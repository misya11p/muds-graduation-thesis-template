\documentclass[titlepage,12pt]{ltjsarticle}

\usepackage{graphicx}
\usepackage{amsmath}
\usepackage{amssymb}
\usepackage{titlesec}
\usepackage{lmodern}
\usepackage{tocloft}
\usepackage[hidelinks]{hyperref}
\usepackage[no-math,deluxe,expert,ipa]{luatexja-preset}

\pretocmd{\section}{\clearpage}{}{}
\titleformat{\section}{\normalfont\Large}{\thesection}{1em}{}
\titleformat{\subsection}{\normalfont\large}{\thesubsection}{1em}{}
\titleformat{\subsubsection}{\normalfont}{\thesubsubsection}{1em}{}

\setcounter{tocdepth}{3}
\cftsetindents{section}{0em}{3em}
\cftsetindents{subsection}{1em}{3em}
\cftsetindents{subsubsection}{2em}{3em}

\title{卒業論文 \vskip\baselineskip タイトルタイトルタイトル}
\author{武蔵野太郎\\ データサイエンス学科\\ 武蔵野大学}
\date{2025年1月}

\begin{document}

% Title Page
\maketitle

% Abstract
\section*{要旨}
「ダミーテキスト」近年,プロの作家やライターではない,...

% Table of Contents
\clearpage
\tableofcontents

% section 1: Introduction
\section{序論}
このファイルは,武蔵野大学データサイエンス学部データサイエンス学科の卒論のテンプレートです....

\subsection{論文の構成}
論文の構成を以下に述べる.2章では...

% section 2: Figures, Tables, and Equations
\section{図と表と数式}
\subsection{図表の配置}
図表は本文中に,出現する場所に配置します.

\subsection{図表のキャプション}
図と表にはキャプションを付けます.キャプションというのは...

\subsubsection{図とそのキャプションの例}
\begin{figure}[htbp]
    \centering
    \includegraphics[width=0.7\textwidth]{example-image-a}
    \caption{人間の知識創造のベクトル空間モデルにおける表現}
\end{figure}

\subsubsection{表とそのキャプションの例}
\begin{table}[htbp]
    \centering
    \begin{tabular}{|c|c|c|c|}
        \hline
        ML & n(特徴) & m(印象) & 性質 \\ \hline
        楽曲 (Hevner) & 6 & 8 & 特徴保存 \\ \hline
        カラーイメージスケール & 180 & 130 & 印象保存 \\ \hline
        音相理論 & 78 & 20 & 印象保存 \\ \hline
    \end{tabular}
    \caption{各MLの性質}
\end{table}

\subsection{数式の参照番号}
\begin{equation}
    S_1 = f(s_i) \quad 1 \leq i \leq n
\end{equation}
...

% section 3: Hierarchy of Headings and Style
\section{見出しの階層構造とスタイルの設定}
\subsection{見出しのレベル}
\subsubsection{これはレベル3です}
...

% section 4: Conclusion
\section{結論}
本論文では,...

% Acknowledgements
\section*{謝辞}
本研究の遂行にあたって,...

% References
\bibliography{references}

\end{document}
